\documentclass{article}
\usepackage[utf8]{inputenc}
\usepackage[T1]{fontenc}
\usepackage{csvsimple}
\usepackage{rotating}
\usepackage{graphicx}
\usepackage{amsmath}
\usepackage{listings}
\setlength{\parindent}{0pt}
\title{Symulacje Monte Carlo}
\date{}

\begin{document}

\maketitle
Temat: \textbf{Model Isinga 2D}\\
Imię i nazwisko prowadzącego: \textbf{Grzegorz Pawlik}

\begin{center}
\begin{tabular}{|p{5cm}|p{6cm}|}
\hline
$Wykonawca:$ & $Gracjan$ $Tokarz$ $255531$ $W11$  \\
\hline
$Termin$ $zajec:$ & $Piatek , 15:15$\\
\hline
$Data$ $oddania$ $sprawozdania:$ & $15.01.2021r$\\
\hline
\textbf{Ocena	końcowa:} & \\
\hline
\end{tabular}
\end{center}

\textbf{Adnotacje dotyczące wymaganych poprawek,oraz daty otrzymania poprawionego sprawozdania}

\newpage

\section{Kod źródłowy}

\lstset{
	language=C++
}
\begin{lstlisting}[basicstyle=\small]

#include <stdio.h>
#include <stdlib.h>
#include <string.h>
#include <math.h>

int main(int argc, char* argv[]){
	long int steps = 230000;
	//long int steps = 2300000;
	int skip = 30000;
	int interval = 100;

	float T = atof(argv[1])/10;
	float R;
	int L = atoi(argv[2]);
	int S[L][L], P[L], N[L];
	for(int i = 0; i < L; i++){//nearest neighbor
		N[i] = i + 1;//next
		P[i] = i - 1;//previous
	}
	P[0] = L-1;
	N[L-1] = 0;

	for(int i = 0; i < L; i++){//initial config
		for(int j = 0; j < L; j++){
			S[i][j] = 1;
		}
	}

	int dE; 
	float omega, m = 0, mTemp, usedConfigs = (steps-skip)/interval;
	for(int i = 1; i <= steps; i++){//main loop
	for(int j = 0; j < L; j++){
		for(int k = 0; k < L; k++){
			dE = 2*S[j][k]*(S[N[j]][k]+S[P[j]][k]+S[j][N[k]]+S[j][P[k]]);
			if(dE < 0){
				S[j][k] = -S[j][k];
			}
			else{
				omega = exp(-dE/T);
				R = (float)rand()/RAND_MAX;
				if(R < omega){
					S[j][k] = -S[j][k];
				}
			}
		}
	}
	if(atoi(argv[3]) == 1){//getting config
		FILE *fptr;
		fptr = fopen(argv[4],"w");
		for(int i = 0; i < L; i++){
			for(int j = 0; j < L; j++){
				fprintf(fptr,"%d\t",S[i][j]);
			}
			fprintf(fptr,"\n");
		}
		fclose(fptr);
	}	
return 0;

\end{lstlisting}

\newpage
\section{Wyniki obliczeń i parametry symulacji}
\begin{center}
\begin{tabular}{p{5cm}|p{5cm}}
	Ilość kroków MC: & $230 000$\\
	\hline
	MCS przy obserwacji przeskoków & $2 300 000$\\
	\hline
	MCS pominięte przy średnich & $30 000$\\
	\hline
	Interwał odczytywanych konfiguracji& $100$\\
	\hline
	Początkowy spin & $1$\\
	\hline
	$T^*_C$ & $2.269$\\
	\hline
	$\beta$ & $0.125$\\
	\hline
	$\nu$ & $1$\\
	\hline
	Krok T* & $0.1$\\
	\hline
	Krok T* w okolicach $T_C$ & $0.02$\\
	\hline
	Krok T* w okolicach $T_C$ (L=80)& $0.01$\\
\end{tabular}
\end{center}

\section{Wykresy}
\subsection{Przeskoki magnetyzacji na wykresie m(T*)}
\includegraphics[width=13cm]{flips3.png}

\subsection{Konfiguracja układu dla L = 5, T* = 0.1}
\includegraphics[width=13cm]{conf5-01.png}
\subsection{Konfiguracja układu dla L = 5, T* = 2.269 = Tc}
\includegraphics[width=13cm]{conf5-tc.png}
\subsection{Konfiguracja układu dla L = 5, T* = 10}
\includegraphics[width=13cm]{conf5-10.png}

\subsection{Konfiguracja układu dla L = 20, T* = 0.1}
\includegraphics[width=13cm]{conf20-01.png}
\subsection{Konfiguracja układu dla L = 20, T* = 2.269 = Tc}
\includegraphics[width=13cm]{conf20-tc.png}
\subsection{Konfiguracja układu dla L = 20, T* = 10}
\includegraphics[width=13cm]{conf20-10.png}

\subsection{Konfiguracja układu dla L = 80, T* = 0.1}
\includegraphics[width=13cm]{conf80-01.png}
\subsection{Konfiguracja układu dla L = 80, T* = 2.269 = Tc}
\includegraphics[width=13cm]{conf80-tc.png}
\subsection{Konfiguracja układu dla L = 80, T* = 10}
\includegraphics[width=13cm]{conf80-10.png}

\subsection{Zależność średiej magnetyzacji od temperatury zredukowanej i rozmiaru}
\includegraphics[width=13cm]{MofT.png}

\subsection{Teoria skalowania z rozmiarem}
\includegraphics[width=13cm]{scale.png}

\end{document}
